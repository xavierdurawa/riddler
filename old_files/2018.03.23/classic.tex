\documentclass[a4paper,12pt]{article}
\usepackage{amsfonts}
\usepackage{amsmath}
\begin{document}


\section{Problem}

"Imagine taking a number and moving its last digit to the front.
For example, 1,234 would become 4,123. What is the smallest positive 
integer such that when you do this, the result is exactly double the 
original number? (For bonus points, solve this one without a 
computer.)"

\section{Solution}

Let's begin by writing our numbers in the following form\smallskip

\begin{equation}
\begin{aligned}
\label{eq:5} & 10^Nd_N+10^{N-1}d_{N-1}+\cdots+10^1d_1+10^0d_0 = 
\sum^N_{n=0}10^nd_n \\ & \mbox{ where }d_n \in \{0,1,2,\ldots,9\}
\end{aligned}
\end{equation}
\smallskip

\noindent We are looking for a number that satisfies the following 
equality\smallskip

\begin{equation}
2\sum^N_{n=0}10^nd_n=10^Nd_0+\sum^N_{n=1}10^{n-1}d_n\smallskip
\end{equation}

\noindent Rearranging we get\smallskip

\begin{equation}
\sum^N_{n=1}(2 \cdot 10^n-10^{n-1})d_n=(10^N-2 \cdot 10^0)d_0\smallskip
\end{equation}

\noindent Note that $2 \cdot 10^n-10^{n-1}=19 \cdot 10^{n-1}$.  As such we 
rewrite the equation as follows\smallskip

\begin{equation} \label{eq:3}
\sum^N_{n=1}10^{n-1}d_n=\frac{(10^N-2)d_0}{19}\smallskip
\end{equation}

\noindent Since we know that the left side of the equation is an integer, we 
also know that the numerator of the right side has to be divisible by 19!  
As such, we know that one of the following must be true:\smallskip

\begin{equation} \label{eq:1}
10^N-2 \equiv 0 \ (mod \ 19)\smallskip
\end{equation}

\noindent \textit or \smallskip

\begin{equation} \label{eq:2}
d_0 \equiv 0 \ (mod \ 19)\smallskip
\end{equation}

\noindent Because we previously defined $d_n \in \{0,1,2,\ldots,9\}$, we
know that equation \ref{eq:2} is only true when $d_0=0$ but this solution is 
trivial. Therefore, we will impose equation \ref{eq:1} which we can solve 
for $N$ to get\smallskip

\begin{equation}
N=18n+17 \mbox{ where } n \in \mathbb{Z}_{\geq 0}\smallskip
\end{equation}

\noindent The smallest value that satisfies this equation is $n=0$ which 
yields $N=17$.  Because out indexing begins at 0, we can deduce that the 
smallest possible number that satisfies the problem is 18 digits!  Once we 
set $N=17$ we observe that the right side of equation \ref{eq:3} becomes
\smallskip

\begin{equation}
\sum^N_{n=1}10^{n-1}d_n=\frac{(10^{17}-2)d_0}{19}\smallskip
\end{equation}

\noindent We know see that the left side of the equation can only take one 
of 10 different values (i.e. each of the 10 possible values for $d_0$) and 
represents the first 17 digits of the solution.  Starting from 0, we find 
that $d_0=2$ is the first value that satisfies equation \ref{eq:5}.  
This is shown below\smallskip

\begin{equation}
\sum^N_{n=1}10^{n-1}d_n=1026315789473684\smallskip
\end{equation}

\noindent Appending $d_0$ to the end of this number yields the following
number and the solution to the riddle.\smallskip  

\begin{equation}
10263157894736842\smallskip
\end{equation}

\noindent Move the 2 to the front of the number\smallskip

\begin{equation}
21026315789473684\smallskip
\end{equation}

\noindent and\smallskip

\begin{align*}
2 \ \cdot \ & 1026315789473684 \ 2 \\
    = \ 2 \ & 1026315789473684
\end{align*}


\end{document}
